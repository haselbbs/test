%macropackage=lplaing

\documentstyle[bibgerm]{article}

\def\tex{{\the\textfont0\TeX}}
\def\pLaTeX{{\the\textfont0 L\kern-.36em\raise.3ex\hbox{\the\scriptfont0 
A}\kern-.15em    T\kern-.1667em\lower.7ex\hbox{E}\kern-.125emX}}
\def\LaTeX{\protect\pLaTeX}
\def\pBibTeX{{\the\textfont0 B\hbox{\the\scriptfont0 IB}\kern-.1em\TeX}}
\def\BibTeX{\protect\pBibTeX}
\def\bibtex{\BibTeX}\def\tex{\TeX}
\def\latex{\LaTeX}
\def\dat#1{{\tt #1}}
\def\bsl{{\tt\char'134}}   % '\'
\def\leerz{{\tt\char'040}} % little 'u' for spaces
\def\affe{\char'100}       % '@'
\exhyphenpenalty=50 \hyphenpenalty=10 \doublehyphendemerits=50
\finalhyphendemerits=50 \brokenpenalty=100 \pretolerance=0 \tolerance=2000
\looseness=0 \linepenalty=10
\hoffset=-1.0in \textwidth6.4in \oddsidemargin0.8in \evensidemargin0.8in
\marginparwidth0.6in \marginparsep0.2in \marginparpush1ex
\topmargin0pt \headheight2.5ex \headsep5ex \textheight9.25in
\columnsep10pt \columnseprule0pt
\parindent0pt \parskip2.5ex plus1.5ex minus1.5ex
\renewcommand{\arraystretch}{1.2}
\renewcommand{\baselinestretch}{1.1}
\def\footnoterule{\vfill\kern-3pt \hrule width \textwidth\kern 2.6pt}
\pagestyle{headings}
\newlength{\mylena}
\originalTeX\nonfrenchspacing

\begin{document}

\title{German and international \bibtex ing\\A manual for the 'german' \bibtex\ styles}
\author{Martin Wallmeier\thanks{Tel.: 
[[0]0\,49]\,(09\,31)\,80\,05\,--\,5\,10}\\ G"obelslehenstr. 1--b\,16\\8700 
W"urzburg\\ West Germany\\e-mail: neuk121@wrzx13.rz.uni-wuerzburg.de}
\date{20$^{th}$ April 1993}
\maketitle

\tableofcontents

\section{Introduction}

The original \bibtex\ version 0.99b standard styles allow only English 
references. There are some so called \dat{'German'} styles, but they allow 
only German references. I couldn't find any style beeing able to switch 
between the language for mixed references. In scientific publications you 
will find a lot of English references beside the ones of the country, where 
the publication was written. But it looks awful, if you mix an english 
reference with german expressions, for example:
\begin{center}\begin{minipage}{.7\textwidth}\raggedright\tt Wilson, 
Braunwald, und Isselbacher, Herausgeber. Harrison's Principles of Internal 
Medicine. Band 1, twelfth Auf{}lage\ldots\end{minipage}\end{center}
So I tried to write a style being able to switch the language.

\section[Restrictions of and problems with \bibtex]{Restrictions of and 
problems with \bibtex\footnote{Dedicated to Oren Patashnik at Stanford 
University. I hope this will find him anytime.}}

There is a terrific restriction at least in my version of \bibtex: it 
allows only ten (10) global variables of type \dat{STRINGS}. Therefore it 
was impossible to implement a function which would decide whether to switch 
the language and then call a specific function for every country that could 
actualize the contents of these string variables. So the only possible way 
seemed to be to create a series of \tex\ macros.

Another problem occured with \bibtex's built in function 
\verb|add.period$|. This function adds a point, if the last non '\}' 
character is no '.', '!' or '?'. But how to tell \verb|add.period$| that 
the \tex\ macros for abbreviations contain a point, although there is not 
one obvious for \bibtex? The \tex\ macros have one parameter now: If you 
abbreviate editors for example as 'eds.' \verb|add.period$| mustn't set a 
point if the corresponding \tex\ macro is set in the output. So {\tt\bsl 
btxeditorsshort} is defined as follows:\\[.5ex]
\verb|      \def\btxeditorsshort#1{eds#1}|\\[.5ex]
Now you can write {\tt\bsl btxeditorsshort\{.\}} in the \bibtex\ style file 
and \verb|add.period$| will find a point if {\tt\bsl btxeditorsshort\{.\}} 
is the last expression in the sentence. As you can see this is only 
necessary for those expressions that can stand at the end of a sentence or 
block. 'And' usually will not stay at the end of a sentence. So you can 
define the corresponding macros as pointed abbreviation, if you want to.

I had to face a last problem with the built in function 
\verb|change.case$|. It's similar to the one with \verb|add.period$|.  This 
function is also unable to expand the defined \tex\ macros. Therefore I had 
to define always two forms of macros \verb|\btx...| and \verb|\Btx...| 
where the second should be used where something like `The Quick Brown Fox 
Jumps Over The Lazy Dog' is to occure and the first for something like `The 
quick brown fox\ldots' if the expression starts a sentence or `the quick 
brown fox\ldots' if it is used in the middle of a sentence. The {\tt 
change.case\$} function will not be called if a title is going to be 
formatted and {\tt language} is not set to {\tt 'english'} or {\tt 
'USenglish'}. Henceforth you don't have to use '\{\}' braces to avoid {\tt 
chang.case\$} in titles of other languages. Thus typing is much easier. 
Typing comfort could be also improoved very much if \bibtex\ would 
understand umlauts as \tex\ version 3.1 and not only constructions like 
\verb|{\"{a}}|.

All these constructions are tricky and uncomfortable only to bypass or fool 
the built in \bibtex\ functions. Things would be much easier if more global 
string variables could be defined. Then \bibtex\ could format the content 
of the string variables directly and commands like \verb|add.period$| or 
\verb|\cange.case$| would work the same way as if the expression would 
occure in the text.

Now you know about the reasons why I had to make a new \latex\ style 
\dat{bibgerm.sty} for use with the \dat{ger[...].bst} \bibtex\ styles. It 
defines the \tex\ macros mentioned above and listed below.

\section{Implementation}

\bibtex\ was changed in the way that it doesn't place string constants, but 
\tex\ macros into the output for expressions that vary in the different 
languages. So \bibtex\ generates a {\tt\bsl selectlanguage\{...\}} command 
just before the current \verb|\bibitem[...]{...}| if the {\tt language} 
field in the database entry is not empty. There are only a few allowed 
contents for {\tt language}: {\tt 'german'}, {\tt 'austrian'}, {\tt 
'english'}, and  {\tt 'USenglish'}, because I'm speaking only German and 
bad English. {\tt 'english'} does the same as {\tt 'USenglish'} at the 
moment. I think there won't be many differences, but I could imagine some 
concerning 'Master's thesis' and 'PhD thesis'. {\tt 'austrian'} does nearly 
the same as {\tt 'german'}, only the month 'January' is different.

It should be not so very difficult to implement other languages for someone 
who is a little bit acquainted to \tex\ or \latex. You probably needn't 
change the \bibtex\ styles, but only to supplement the \dat{bibgerm.sty} 
\latex\ style a little bit.

\subsection{\ifx\undefined\selectfont \else\series{m}\fi\dat{bibgerm.sty}}

The macro \verb|\selectlanguage{...}| of the original \dat{german.sty} 
\cite{partl}switched already things like {\tt 'date'} and {\tt 
'captionnames'}. There are now some new global definitions 
\verb|\bibsgerman|, \verb|\bibsenglish| and \verb|\bibsUSenglish| being 
called by {\tt\bsl selectlanguage\{...\}}. The \verb|\bibs...| macros 
redefine the meanings for a list of \tex\ macros according to the current 
language:
\settowidth{\mylena}{\tt\bsl def\bsl ...short\#\#1\{...\#\#1\}}
\begin{center}\begin{tabular}{|l|p{.67\textwidth}@{}l|}\hline
\multicolumn{3}{|c|}{General Aspects}\\\hline
\hbox to\mylena{\tex\ macro\hfil} & Meaning &\\\hline
{\tt\bsl def\bsl btx...\{...\}} & \raggedright Expression in lower case 
letters for use in the middle of a sentence (e.\ g.\ 'volume').&\\
{\tt\bsl def\bsl Btx...\{...\}} & \raggedright Expression with the first 
letter capitalized for use at the beginning of a sentence (e.\ g.\ 
'Volume') or formatted like a booktitle (e.\ g.\ 'Technical Report 
037').&\\
{\tt\bsl def\bsl ...short\#\#1\{...\#\#1\}} & \raggedright Mostly 
abbreviated expression. The tricky construction {\tt\#\#1} was explained 
above. You need double {\tt'\#'} because the macros are defined inside the 
global definition of the {\tt\bsl bibs...} macro. The parameter {\tt\#\#1} 
is used to set the last point of an abbreviation that can occure at the end 
of a sentence (e.\ g.\ {\tt\bsl def\bsl btxetalshort\#\#1\{et~al\#\#1\}}). 
If you don't want to have a point at the end you skip the {\tt\#\#1} inside 
the braces.&\\
{\tt\bsl def\bsl ...long\{...\}} & \raggedright Expression used for the 
long i.\,\,e.\ unabbreviated form (e.\ g.\ {\tt\bsl def\bsl btxetallong\{ 
et~alii\}}). &\\\hline
\multicolumn{3}{|c|}{Details}\\\hline
\hbox to\mylena{\tex\ macro\hfil} & meaning &\\\hline
{\tt\bsl def\bsl btxetalshort\#\#1} & \raggedright Used after authors or 
editors, if there is {\tt 'and othrers'} at the end of the corresponding 
\bibtex\ database entry field (defined as {\tt 'et~al\#\#1{}'}). This 
expression might finish a sentence or block; so {\tt '\#\#1'} is necessary 
for {\tt 'add.period\$'}. &\\
{\tt\bsl def\bsl btxetallong} & \raggedright Unabbreviated form of the 
above (defined as {\tt 'et~alii'}). &\\
{\tt\bsl def\bsl btxandshort} & \raggedright Used before the last author or 
editor if there are more than one (defined as {\tt 'and'}). &\\
{\tt\bsl def\bsl btxandlong} & \raggedright Unabbreviated form of the above 
(defined as {\tt 'and'}). &\\
{\tt\bsl def\bsl btxinshort} & \raggedright Used with entry type {\tt 
@incollection} or {\tt @inproceedings} as introduction for the reference to 
the superior work (defined as {\tt 'in'}). &\\
{\tt\bsl def\bsl btxinlong} & \raggedright Unabbreviated form of the above 
(defined as {\tt 'in'}). &\\
{\tt\bsl def\bsl Btxinshort} & \raggedright For use at the beginning of a 
sentence (defined as {\tt 'In'}). &\\
{\tt\bsl def\bsl Btxinlong} & \raggedright Unabbreviated form of the above 
(defined as {\tt 'In'}). &\\
{\tt\bsl def\bsl btxofseriesshort} & \raggedright Used after the volume 
number before the title of the book series (defined as {\tt 'of'}). &\\
{\tt\bsl def\bsl btxofserieslong} & \raggedright Unabbreviated form of the 
above (defined as {\tt 'of'}). &\\
{\tt\bsl def\bsl btxinseriesshort} & \raggedright Used for a part of a book 
with a separate title. If there is a number given it is placed before the 
title of the series (defined as {\tt 'in'}). &\\
{\tt\bsl def\bsl btxinserieslong} & \raggedright Unabbreviated form of the 
above (defined as {\tt 'in'}). &\\
\hline\end{tabular}

\begin{tabular}{|l|p{.67\textwidth}@{}l|}\hline
\hbox to\mylena{\tex\ macro\hfil} & meaning &\\\hline
{\tt\bsl def\bsl btxeditorshort\#\#1} & \raggedright Abbreviation for a 
single {\em editor\/} that can be placed at the end of the editor block 
(defined as {\tt 'ed\#\#1{}'}). &\\
{\tt\bsl def\bsl btxeditorlong} & \raggedright Unabbreviated form of the 
above (defined as {\tt 'editor'}). &\\
{\tt\bsl def\bsl btxeditorsshort\#\#1} & \raggedright Equivalent of the 
above for more than one editor (defined as {\tt 'eds\#\#1{}'}). &\\
{\tt\bsl def\bsl btxeditorslong} & \raggedright Unabbreviated form of the 
above (defined as {\tt 'editors'}). &\\
{\tt\bsl def\bsl Btxeditorshort\#\#1} & \raggedright This and the following 
macros could be used by a style file that would start a bibliography entry 
with something like 'Ed.:\ \ldots' (defined as {\tt 'Ed\#\#1{}'}). &\\
{\tt\bsl def\bsl Btxeditorlong} & \raggedright Unabbreviated form of the 
above (defined as {\tt 'Editor'}). &\\
{\tt\bsl def\bsl Btxeditorsshort\#\#1} & \raggedright Plural of the above 
(defined as {\tt 'Eds\#\#1{}'}). &\\
{\tt\bsl def\bsl Btxeditorslong} & \raggedright Unabbreviated form of the 
above (defined as {\tt 'Editors'}). &\\
{\tt\bsl def\bsl btxvolumeshort\#\#1} & \raggedright Abbreviation for {\em 
volume\/} for use after a comma, i.\,\,e.\ in the middle of a sentence 
(defined as {\tt 'vol\#\#1{}'}). &\\
{\tt\bsl def\bsl btxvolumelong} & \raggedright Unabbreviated form of the 
above (defined as {\tt 'volume'}). &\\
{\tt\bsl def\bsl Btxvolumeshort\#\#1} & \raggedright For use at the 
beginning of a sentence (defined as {\tt 'Vol\#\#1{}'}). &\\
{\tt\bsl def\bsl Btxvolumelong} & \raggedright Unabbreviated form of the 
above (defined as {\tt 'Volume'}). &\\
{\tt\bsl def\bsl btxnumbershort\#\#1} & \raggedright Abbreviation for {\em 
number\/} if used in the middle of a sentence (defined as {\tt 
'no\#\#1{}'}). &\\
{\tt\bsl def\bsl btxnumberlong} & \raggedright Unabbreviated form of the 
above (defined as {\tt 'number'}). &\\
{\tt\bsl def\bsl Btxnumbershort\#\#1} & \raggedright For use at the 
beginning of a sentence (defined as {\tt 'No\#\#1{}'}). &\\
{\tt\bsl def\bsl Btxnumberlong} & \raggedright Unabbreviated form of the 
above (defined as {\tt 'Number'}). &\\
{\tt\bsl def\bsl btxeditionshort\#\#1} & \raggedright Abbreviation for {\em 
edition\/} that can finish a sentence or block (defined as {\tt 
'ed\#\#1{}'}). &\\
{\tt\bsl def\bsl btxeditionlong} & \raggedright Unabbreviated form of the 
above (defined as {\tt 'edition'}). &\\
{\tt\bsl def\bsl Btxeditionshort\#\#1} & \raggedright For use at the 
beginning of a sentence (defined as {\tt 'Ed\#\#1{}'}). &\\
{\tt\bsl def\bsl Btxeditionlong} & \raggedright Unabbreviated form of the 
above (defined as {\tt 'Edition'}). &\\
{\tt\bsl def\bsl btxchaptershort\#\#1} & \raggedright Abbreviation for {\em 
chapter\/} which might be placed at the end of a sentence if it would be 
used after first, second, etc. (defined as {\tt 'chap\#\#1{}'}). &\\
{\tt\bsl def\bsl btxchapterlong} & \raggedright Unabbreviated form of the 
above (defined as {\tt 'chapter'}). &\\
{\tt\bsl def\bsl Btxchaptershort\#\#1} & \raggedright For use at the 
beginning of a sentence (defined as {\tt 'Chap\#\#1{}'}). &\\
{\tt\bsl def\bsl Btxchapterlong} & \raggedright Unabbreviated form of the 
above (defined as {\tt 'Chapter'}). &\\
{\tt\bsl def\bsl btxpageshort\#\#1} & Abbreviation for {\em page\/} 
\raggedright (defined as {\tt 'p\#\#1{}'}). &\\
{\tt\bsl def\bsl btxpagelong} & \raggedright Unabbreviated form of the 
above (defined as {\tt 'page'}). &\\
{\tt\bsl def\bsl btxpagesshort\#\#1} & \raggedright If there are more than 
one {\em page\/} this abbreviation will be used (defined as {\tt 
'pp\#\#1{}'}). &\\
{\tt\bsl def\bsl btxpageslong} & \raggedright Unabbreviated form of the 
above (defined as {\tt 'pages'}). &\\
{\tt\bsl def\bsl Btxpageshort\#\#1} & \raggedright For use at the beginning 
of a sentence (defined as {\tt 'P\#\#1{}'}). &\\
{\tt\bsl def\bsl Btxpagelong} & \raggedright Unabbreviated form of the 
above (defined as {\tt 'Page'}). &\\
{\tt\bsl def\bsl Btxpagesshort\#\#1} & \raggedright Plural of the above 
(defined as {\tt 'Pp\#\#1{}'}). &\\
{\tt\bsl def\bsl Btxpageslong} & \raggedright Unabbreviated form of the 
above (defined as {\tt 'Pages'}). &\\
\hline\end{tabular}

\begin{tabular}{|l|p{.67\textwidth}@{}l|}\hline
\hbox to\mylena{\tex\ macro\hfil} & Meaning &\\\hline
{\tt\bsl def\bsl btxmastthesis} & \raggedright Expression that will be used 
by default if there is given no \bibtex\ database entry field {\tt type} in 
a {\tt @masterthesis} entry (defined as {\tt 'Master's thesis'}). &\\
{\tt\bsl def\bsl btxphdthesis} & \raggedright Equivalent of the above for 
the {\tt @phdthesis} entry (defined as {\tt 'PhD thesis'}). &\\
{\tt\bsl def\bsl btxtechrepshort\#\#1} & \raggedright Equivalent of the 
above for abbreviated {\tt @techreport} entry formatted like the title of 
an article or journal, i.\,\,e.\ only the first letter capitalized (defined 
as {\tt 'Techn.\ rep\#\#1{}'}). &\\
{\tt\bsl def\bsl btxtechreplong} & \raggedright Unabbreviated form of the 
above (defined as {\tt 'Technical report'}). &\\
{\tt\bsl def\bsl Btxtechrepshort\#\#1} & \raggedright Same as the above but 
formatted like a booktitle, i.\,\,e.\ all words capitalized except articles 
and conjunctions like 'the' or 'of', etc. This one is used if the number 
field of the {\tt @techreport} entry is not empty (defined as {\tt 'Techn.\ 
Rep\#\#1{}'}). &\\
{\tt\bsl def\bsl Btxtechreplong} & \raggedright Unabbreviated form of the 
above (defined as {\tt 'Technical Report'}). &\\
{\tt\bsl def\bsl btxmonjanshort\#\#1} & \raggedright Abbreviated names of 
the months (defined as {\tt 'Jan\#\#1{}'}). &\\
{\tt\bsl def\bsl btxmonfebshort\#\#1} & \raggedright (defined as {\tt 
'Feb\#\#1{}'}). &\\
{\tt\bsl def\bsl btxmonmarshort\#\#1} & \raggedright (defined as {\tt 
'Mar\#\#1{}'}). &\\
{\tt\bsl def\bsl btxmonaprshort\#\#1} & \raggedright (defined as {\tt 
'Apr\#\#1{}'}). &\\
{\tt\bsl def\bsl btxmonmayshort\#\#1} & \raggedright (defined as {\tt 
'May'}). &\\
{\tt\bsl def\bsl btxmonjunshort\#\#1} & \raggedright (defined as {\tt 
'June'}). &\\
{\tt\bsl def\bsl btxmonjulshort\#\#1} & \raggedright (defined as {\tt 
'July'}). &\\
{\tt\bsl def\bsl btxmonaugshort\#\#1} & \raggedright (defined as {\tt 
'Aug\#\#1{}'}). &\\
{\tt\bsl def\bsl btxmonsepshort\#\#1} & \raggedright (defined as {\tt 
'Sept\#\#1{}'}). &\\
{\tt\bsl def\bsl btxmonoctshort\#\#1} & \raggedright (defined as {\tt 
'Oct\#\#1{}'}). &\\
{\tt\bsl def\bsl btxmonnovshort\#\#1} & \raggedright (defined as {\tt 
'Nov\#\#1{}'}). &\\
{\tt\bsl def\bsl btxmondecshort\#\#1} & \raggedright (defined as {\tt 
'Dec\#\#1{}'}). &\\
{\tt\bsl def\bsl btxmonjanlong} & \raggedright Unabbreviated names of the 
months (defined as {\tt 'January'}). &\\
{\tt\bsl def\bsl btxmonfeblong} & \raggedright (defined as {\tt 
'February'}). &\\
{\tt\bsl def\bsl btxmonmarlong} & \raggedright (defined as {\tt 'March'}). 
&\\
{\tt\bsl def\bsl btxmonaprlong} & \raggedright (defined as {\tt 'April'}). 
&\\
{\tt\bsl def\bsl btxmonmaylong} & \raggedright (defined as {\tt 'May'}). 
&\\
{\tt\bsl def\bsl btxmonjunlong} & \raggedright (defined as {\tt 'June'}). 
&\\
{\tt\bsl def\bsl btxmonjullong} & \raggedright (defined as {\tt 'July'}). 
&\\
{\tt\bsl def\bsl btxmonauglong} & \raggedright (defined as {\tt 'August'}). 
&\\
{\tt\bsl def\bsl btxmonseplong} & \raggedright (defined as {\tt 
'September'}). &\\
{\tt\bsl def\bsl btxmonoctlong} & \raggedright (defined as {\tt 
'October'}). &\\
{\tt\bsl def\bsl btxmonnovlong} & \raggedright (defined as {\tt 
'November'}). &\\
{\tt\bsl def\bsl btxmondeclong} & \raggedright (defined as {\tt 
'December'}). &\\
\hline\end{tabular}\end{center}

For more details about macro programming refer to \cite{texbook}.

\subsection{Available \bibtex\ styles and file list}

All \bibtex\ standard styles are available although I've changed their 
design a little bit. Furthermore, I adjusted the \dat{'apalike.bst'} which 
allows references like '(Jones 1986)'. The names of the 'german' \bibtex\ 
styles, their corresponding standards, and the remaining files are listed 
in the following table:
\begin{center}\begin{tabular}{|l|l|}\hline
\multicolumn{2}{|c|}{\bibtex\ Styles and Files}\\\hline
'german' & Standard \\\hline
{\tt gerabbrv.bst} & {\tt abbrv.bst} \\
{\tt geralpha.bst} & {\tt alpha.bst} \\
{\tt gerplain.bst} & {\tt plain.bst} \\
{\tt gerunsrt.bst} & {\tt unsrt.bst} \\\hline
{\tt gerapali.bst} & {\tt apalike.bst} \\\hline
{\tt gerxampl.bib} & {\tt xampl.bib} \\\hline\hline
\multicolumn{2}{|c|}{\latex\ Styles}\\\hline
'german' & Standard \\\hline
{\tt bibgerm.sty} & {\tt german.sty} \\
{\tt apalike.sty} & {\tt apalike.sty} \\\hline\hline
\multicolumn{2}{|c|}{Manual for \latex}\\\hline
{\tt gerbibtx.tex} & {\sc Manual} and \\
{\tt gerbibtx.bib} & {\sc Bibliography}\\\hline\hline
\multicolumn{2}{|c|}{Original Manuals}\\\hline
{\tt btxdoc.tex} & For general users\\
{\tt btxhak.tex} & For style designers\\
{\tt btxdoc.bib} & {\sc Bibliography}\\\hline
\end{tabular}\end{center}

If you want to write \bibtex\ styles on your own, you are absolutely in 
need of \cite{bibtex.a} and \cite{bibtex.b}. These documents should be as 
files in your distribution of \bibtex. If they are not, complain violently 
to the distributor.

\subsection{Changes to the standard styles}

There are two new functions: {\tt after.authors} and {\tt smallcaps}. The 
last is similar to {\tt emphasize}. It sets the authors or editors in {\sc 
Small Caps}. If you don't like that, you can change it to an empty 
function. The other one sets the new 'state' {\tt before.title}. These 
'states' are used by the different {\tt output} routines to decide what 
punctuation is going to be placed (most times point or comma) before the 
string on the stack. If the current 'state' is {\tt before.title} a colon 
is placed before the title of what ever, i.\,\,e. regularly after authors 
and editors. If you don't like this either, you can copy the function {\tt 
new.block} to {\tt after.authors}. Then a point will be set instead of a 
colon. {\tt gerapali.bst} behaves in this way, because a colon after braces 
looks a little bit strange.

I've changed also the format of the authors' and editors' names to my 
personal favorites:

\begin{center}\begin{minipage}{.7\textwidth}\raggedright\tt Wallmeier, M. 
van, O. Patashnik, and L. van Lamport (eds.): \bibtex ing is 
great\ldots\end{minipage}\end{center}

You can reset the original status by comparing the lines where the function 
{\tt format.name\$} occures with the according lines in the original 
standard \bibtex\ styles. Nevertheless you should be familar with \bibtex's 
name formatting \cite[section 5.4]{bibtex.b}.

I'm not in access to the original \latex\ standard work \cite{lamport}. But 
\cite[appendix B]{kopka} is marked as an almost extensive translation from 
\cite{lamport}. If this is true, there are a few differences concerning the 
optional fields in some entry types. They concern the additional field {\tt 
number} that is used by the entry types {\tt @book}, {\tt @inbook}, {\tt 
@incollection}, {\tt @inproceedings}, and {\tt @proceedings} and the 
additional field {\tt tpye} that is used in {\tt @masterthesis} and {\tt 
@phdthesis}.

Those were of most importance for writing {\tt bibgerm.sty} and the {\tt 
ger[...].bst} files, but there are some more \cite[section 2.2 'Changes to 
the standard styles']{bibtex.a}.


\section{Changes}

\begin{description}
\item[1993, April 20th:] \verb|bibgerm.sty| changed for use with `german.sty' 
version 2.4. In detail: redefine `\verb|\selectlanguage|' or 
`\verb|\p@selectlanguage|' depending on version of `german.sty'. 
\end{description}

\section{Known bugs}

\begin{description}
\item[german.sty:] Probably `\verb|bibgerm.sty|' won't work with `german.sty' 
version 3.
\end{description}

\bibliography{gerbibtx}\bibliographystyle{gerplain}

\end{document}

